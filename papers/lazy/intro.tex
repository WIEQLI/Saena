\section{Introduction}\label{sec:intro} %1page

Chemical transport is a highly interesting phenomenon in a variety of fields like environmental pollutants tracking, biological processes like blood clotting and industrial applications like solvent manufacturing. To study this effectively, we need accurate models that track and predict these chemicals. Many of these applications often involve more than one chemical species which move and interact among themselves. As such, modeling of these processes highly depend on the accuracy with which we can track the interplay between these chemicals. Variation in the chemical population often occurs because of movement and reactions among themselves, thus creating more chemical species or destroying the existing ones. Due to the complex nature of the problem, the advection-diffusion equation shown in \eqref{eqn1} is often used to obtain a numerical approximation of the exact location and population of the chemical species. Here, $D$ referes to the diffusion coefficient and $U$ refer to the flow velocity. It is equally important that we track the accuracy of these models while tracking the chemical species. One of the most common way to ensure accuracy is to compare the numerical solution against the analytical solution. The advection-diffusion equation is being extensively studied and analytically solved for the case of constant diffusion coefficient, with uniform flow \cite{carslaw1959heat,van1982analytical}. However, many naturally occurring processes are more involved than the constant diffusion coefficient or uniform flow can model. One such process is described in the next section.


\begin{align}
    \frac{\partial  \vc}{\partial t} = -U\frac{\partial \vc}{\partial x} + D\frac{\partial^2 \vc}{\partial x^2} \label{eqn1}
\end{align}

\subsection{Variable Diffusion case in Elliptic PDE} \label{sec:Thrombosis}

We now build the model problem of variable diffusion which is an example of elliptic PDE, and motivate the application of multigrid preconditioning methods to numerically solve the resulting equation. Consider the set of equation that arise out of the modeling of chemical transport and interactions resulting from the biological process of blood clotting (specifically, \textit{thrombosis}), popularly knows as the Leiderman-Fogelsen model \cite{leiderman2011grow,l2013influence}.\\

Depending on the roles they play in the process, the interesting quantities of this model can be divided into the following classes: 
\begin{enumerate}
\item Mobile unactivated ($P^{m,u}$), 
\item Mobile activated ($P^{m,a}$), 
\item Platelet-bound activated ($P^{b,a}$)
\item Subendothelium-bound activated ($P^{se,a}$)
\end{enumerate}


\begin{align}
    \frac{\partial P^{m,u}}{\partial t} = \underbrace{-\nabla \cdot \{ W(\phi^T) (\vu P^{m,u} - D \nabla P^{m,u}))\} }_\text{Transport by advection and diffusion} \nonumber\\
    - \underbrace{k_{adh}(\vx) \{P_{max} - P^{se,a} \} P^{m,u}}_\text{Adhesion to subendothelium} \nonumber\\
    - \underbrace{\{A_1 (e_2) +A_2([ADP])\}P^{m,u} }_\text{Activation by thrombin or ADP} \label{eqn2}
\end{align}

\begin{align}
    \frac{\partial P^{m,a}}{\partial t} = - \nabla \cdot \{W(\phi^T)(\vu P^{m,a} - D \nabla P^{m,a} ) \} \nonumber\\
    -k_{adh}(\vx)\{P_{max} - P^{se,a} \} P^{m,a} \nonumber\\
    +\{ A_1 (e_2) + A_2([ADP])P^{m,u} \nonumber\\
    -\underbrace{k_{coh} g(\eta) P_{max} P^{m,a}}_\text{Cohesion to bound platelets}. \label{eqn3}
\end{align}
$k_{adh}(\vx)$ is assumed to  be a positive constant for points $\vx$ within one platelet diameter of subendothelium and zero elsewhere. 

\begin{align} 
    W(\phi^T) = tanh(\pi (1-\phi^T)),  \label{eqn6}
\end{align}
where, $\phi^T = P^{se,a}+P^{b,a}+\frac{P_0}{P_{maxse}}(P^{m,u}+P^{m,a})$, $P_{maxse} and P_{max}$ are a constants for maximum number density for platelets as per \cite{leiderman2011grow}.


In order to numerically solve the model equations of the kind \eqref{eqn2} and \eqref{eqn3}, we need to vary the diffusion coefficient  (depicted by $W(\phi^T)D$) per time-step. The resulting elliptical PDEs can be solved using the Finite Element method. We use the Finite Element software package Nektar++ \cite{nekpp} version 4.4.1 to solve the continuous galerkin problem arising out of the changing diffusion coefficient per time-set. It provides an unsteady diffusion solver that expects the mesh file describing the geometry of the domain and the related solver parameters.\\

While multigrid methods, specifically algebraic multigrid (AMG) methods in conjunction with Krylov methods are very effective in solving equations \eqref{eqn2} and \eqref{eqn3}, the high setup cost associated with AMG makes it less attractive when the operator changes very rapidly (due to varying diffusion coefficient). In this work, we explore different strategies for mitigating the high setup cost, while still retaining the efficiency of multigrid. We find that by performing lazy updates for the AMG preconditioner, we can amortize the high setup costs without adversely affecting the convergence rate. We discribe our methods in the next section \S\ref{sec:method} followed by experiments demonstrating the effectiveness of our approach in \S\ref{sec:results}. Finally, we conclude with directions for future research in \S\ref{sec:conclude}. 
